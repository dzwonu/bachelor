\chapter{Teoria}\label{chap:teoria}

\section{Systemy ekspertowe}

\paragraph{}
Według Jana Mulawki:\\System ekspertowy jest programem komputerowym, który wykonuje złożone zadania o~dużych wymaganiach intelektualnych i~robi to tak dobrze jak człowiek, będący ekspertem w~tej dziedzinie.\footnote{\cite{mulawka}, str. 20}\\
Biorąc pod uwagę powyższe można powiedzieć, że~system ekspertowy to aplikacja przeprowadzająca proces wnioskowania na podstawie zbioru specjalistycznej wiedzy. Wnioskowanie takie wymaga oprócz wiedzy, zbioru reguł, według których proces ten będzie przeprowadzany. Można wyróżnić następujące elementy składowe systemu ekspertowego:
\begin{itemize}
	\item Baza wiedzy - Baza specjalistycznej wiedzy (np. zbiór reguł) z~dziedziny, dla~której tworzony jest system ekspertowy
	\item Baza danych stałych - Zbiór danych na temat analizowanego problemu
	\item Silnik wnioskujący - Mechanizm realizujący proces wnioskowania
\end{itemize}

Ponadto, program będący systemem ekspertowym może posiadać dodatkowo następujące elementy:
\begin{itemize}
	\item Interfejs użytkownika - Graficzny bądź tekstowy interfejs dla użytkownika korzystającego z~programu
	\item Edytor wiedzy - Edytor pozwalający rozszerzać bazę wiedzy
	\item Silnik objaśniający - Mechanizm wyjaśniający proces wnioskowania. Prezentuje "tok myślenia", którym system doszedł do~uzyskanych wniosków
\end{itemize}

Już pod koniec lat siedemdziesiątych XX wieku zauważono, że największy wpływ na efektywność działania systemu ekspertowego ma baza wiedzy. Stwierdzono, że~im pełniejsza wiedza, tym szybszy jest proces uzyskiwania wniosków przez program. Stworzenie dobrego systemu ekspertowego opiera się~więc w~dużej mierze na dostarczeniu dla niego odpowiednio dużej bazy dobrej jakościowo wiedzy. Jest to zadanie niełatwe, ponieważ wymaga pozyskania wiedzy od eksperta (bądź grupy ekspertów), który decyzje podejmuje na podstawie informacji o~problemie, ale~również w~oparciu o~swoje doświadczenie. Dodatkowo, zebraną wiedzę należy ustrukturalizować, tak~aby była możliwa do przetwarzania przez system.\footnote{\cite{mulawka}, str. 21}

Systemy ekspertowe możemy podzielić ze względu na kilka kryteriów, jednak ogólnie dzielą się one na:
\begin{itemize}
	\item Doradcze - prezentują rozwiązania dla użytkownika, który potrafi ocenić ich jakość
	\item Podejmujące decyzję bez kontroli człowieka - system sam podejmuje decyzję, nie poddaje rozwiązania ocenie użytkownika
	\item Krytykujące - system ma przedstawiony problem i jego rozwiązanie, a następnie ocenia to rozwiązanie
\end{itemize}
Możemy również podzielić je na dwie grupy:
\begin{itemize}
	\item Systemy dedykowane - kompletny system ekspertowy posiadający bazę wiedzy
	\item Systemy szkieletowe - system z~pustą bazą wiedzy
\end{itemize}
W drugim przypadku proces tworzenia systemu ekspertowego polega na dostarczeniu bazy wiedzy.
Innym kryterium podziału jest ze względu na logikę wykorzystywaną podczas prowadzenia procesu wnioskowania:
\begin{itemize}
	\item Z logiką dwuwartościową (Boole'a)
	\item Z logiką wielowartościową
	\item Z logiką rozmytą
\end{itemize}
Można również dokonać podziału ze~względu na rodzaj przetwarzanej informacji:
\begin{itemize}
	\item Systemy z wiedzą pewną
	\item Systemy z wiedzą niepewną
\end{itemize}
Najbardziej szczegółowy podział to podział ze~względu na zadania realizowane przez system ekspertowy:
\begin{itemize}
	\item Interpretacyjne - dedukują opisy sytuacji z~obserwacji lub stanu czujników
	\item Predykcyjne - wnioskują o~przyszłości na podstawie danej sytuacji
	\item Diagnostyczne - określają wady systemu na podstawie obserwacji
	\item Kompletowania - konfigurują obiekty w~warunkach ograniczeń
	\item Planowania - podejmują działania, aby osiągnąć cel
	\item Monitorowania - porównują obserwacje z~ograniczeniami
	\item Sterowania - kierują zachowaniem systemu
	\item Poprawiania - podają sposób postępowania w~przypadku złego funkcjonowania obiektu
	\item Naprawy - harmonogramują czynności przy dokonywaniu napraw uszkodzonych obiektów
	\item Instruowania - systemy doskonalenia zawodowego (np.~dla studentów)
\end{itemize}

Jak już wcześniej wspomniano, najistotniejszą częścią systemu ekspertowego jest baza wiedzy. Systemy ekspertowe których baza wiedzy składa się z~faktów i~reguł nazywa się systemami regułowymi. Zdecydowana większość systemów ekspertowych to właśnie systemy regułowe. Fakty w~bazie wiedzy są to zdania oznajmujące, opisujące jakiś obiekt bądź jego stan. Reguły natomiast są zdaniami warunkowymi. Gdy spełnione są przesłanki występujące w~części warunkowej takiego zdania, wtedy do bazy faktów dodawane jest nowe zdanie.

Systemy ekspertowe są to aplikacje mieszczące się w~dziedzinie sztucznej inteligencji. Jest to nauka definiowana na wiele sposobów. Dwie definicje prezentujące odmienne aspekty badań prowadzonych w tej dziedzinie:
\begin{itemize}
	\item Według Minsky'ego sztuczna inteligencja jest nauką o maszynach realizujących zadania, które wymagają inteligencji wówczas, gdy są wykonywane przez człowieka.\footnote{\cite{mulawka}, str. 17}
	\item Według Feigenbauma sztuczna inteligencja stanowi dziedzinę informatyki dotyczącą metod i~technik wnioskowania symbolicznego przez komputer oraz symbolicznej reprezentacji wiedzy stosowanej podczas takiego wnioskowania.\footnotemark[3]
\end{itemize}


\section{Programowanie funkcyjne}

\section{Analiza techniczna}