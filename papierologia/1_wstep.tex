\chapter{Wstęp}
Niniejsza praca dotyczy inżynierii oprogramowania i systemów ekspertowych. W wyniku pracy powstał system ekspertowy do oceny spółek giełdowych za pomocą analizy technicznej. Temat ten został podjęty ze względu na zainteresowanie analizą rynków finansowych, głównie rynku akcji, a także chęć poszerzenia wiedzy z zakresu budowy systemów ekspertowych. Realizacja systemu została wykonana z wykorzystaniem podejścia funkcyjnego do tworzenia oprogramowania.

\section{Cele pracy}\label{sec:cele_pracy}

W pracy stawiam następujące cele:

\begin{itemize}
 \item Stworzenie systemu ekspertowego oceniającego spółki giełdowe przy pomocy analizy technicznej
 \item Wykazanie skuteczności powstałego systemu przez analizę porównawczą wyników generowanych przez aplikację i rzeczywistych zachowań cen spółek
 \item Opis zastosowania programowania funkcyjnego w procesie wytwarzania systemu ekspertowego
\end{itemize}

\section{Przegląd literatury oraz uzasadnienie wyboru tematu}

Temat analizy technicznej rynków finansowych rozwija się od przeszło 100 lat i wiedza z tej dziedziny cały czas jest rozszerzana. W obecnych czasach bardzo trudno jest jednej osobie, nawet jeśli jest ekspertem, przeprowadzić rzetelnie analizę techniczną wybranego rynku finansowego bądź aktywa na tym rynku w czasie, który pozwalałby na wykorzystanie wniosków z takiej analizy do wykonania pożądanych operacji na rynku. Stąd też potrzeba posiadania narzędzia, które będzie analizować przykładowo spółki giełdowe w czasie znacząco krótszym niż zrobiłby to człowiek, a przy tym z taką samą bądź lepszą trafnością przewidywań zachowań cen. W mojej pracy podstawowym źródłem wiedzy na temat analizy technicznej jest książka "Analiza techniczna rynków finansowych" \cite{analiza} autorstwa John'a J. Murphy'ego. Wcześniej wspomnianym narzędziem wykonującym analizę z powodzeniem może być system ekspertowy. Taki system może za użytkownika wykonać obliczenia dla wszystkich potrzebnych wskaźników, a także wykorzystując posiadaną bazę wiedzy i reguł przeanalizować zależności pomiędzy wyliczonymi wartościami. Fundamentalną pozycją z zakresu budowy systemów ekspertowych jest pozycja "Systemy ekspertowe" \cite{mulawka} polskiego autora Jana Mulawki. Swój system ekspertowy do analizy technicznej postanowiłem stworzyć z wykorzystaniem funkcyjnego podejścia do programowania. Wybór taki podyktowany był chęcią poszerzenia swojej wiedzy z zakresu metodologii programowania, a także możliwością porównania wybranego podejścia z podejściem klasycznym. Do pisania aplikacji wykorzystałem jeden z dialektów języka Lisp, język Clojure, który również poddam w swojej pracy podstawowej analizie. Więdzę na temat języka Clojure czerpałem przede wszystkim z książki "Programming Clojure" \cite{clojure} napisanej przez Stuart'a Halloway'a oraz Aaron'a Bedra.

\section{Układ pracy}

Struktura dalszej części pracy jest następująca: Rozdział \ref{chap:teoria} zawiera opis teorii ... Rozdział \ref{chap:nowa_teoria} przedstawia nową teorię wprowadzoną przez autora pracy ... Rozdział \ref{chap:narzedzia} opisuje technologie i~narzędzia wykorzystane w~pracy ...  Rozdział \ref{chap:badania} przedstawia wyniki badań / opis stworzonej aplikacji ... Rozdział \ref{chap:podsumowanie} podsumowuje uzyskane wyniki oraz płynące z~nich wnioski ... W~Dodatku \ref{app:edycja} zawarto uwagi dotyczące formatowania pracy z~użyciem systemu \LaTeX. Dodatek \ref{app:plyta} zawiera płytę CD z~aplikacją stworzoną w~ramach pracy...