\chapter{Wyniki badań eksperymentalnych}\label{chap:badania}

\section{Opis stworzonej aplikacji}

\subsection{Podstawowy opis}

\paragraph{}
Aplikacja powstająca podczas pisania tej pracy to system ekspertowy, którego zadaniem jest ocena spółek giełdowych i~podejmowanie decyzji, czy w~danej chwili konkretną spółkę warto kupić bądź sprzedać. Użytkownik aplikacji może pobrać notowania spółek giełdowych notowanych na Giełdzie Papierów Wartościowych w~Warszawie z~serwisu bossa.pl. Może również korzystać z~dowolnego innego źródła notowań, pod warunkiem, że~dane będą archiwum ZIP zawierającym pliki w~formacie MST o~nazwach odpowiadających nazwom spółek. Po tym na liście dostępne są spółki, których dane znajdowały się w~archiwum. Aplikacja po wyborze konkretnej spółki tworzy wykres jej ceny i~wolumenu w~cały okresie w~jakim była notowana. Użytkownik ma możliwość wyboru z~listy ile sesji chce wyświetlić na wykresie, a~także czy ma być tworzony wykres wartości wolumenu. Po wczytaniu danych o~konkretnej spółce przycisk Analizuj pozwala na przeprowadzenie procesu wnioskowania dla danej spółki i~określenie czy w~danej chwili warto sprzedawać bądź kupować dane akcje.

\subsection{Struktura projektu}

\paragraph{}
Projekt podzielony jest na 3 przestrzenie nazw:
\begin{itemize}
	\item bachelor.core
	\item bachelor.wsk
	\item bachelor.zip
\end{itemize}

\begin{tabular}{ c p{10,5cm}}
Przestrzeń nazw & Funkcjonalności \\ \hline \\
bachelor.core & Zawiera wszystkie funkcje parsera reguł, funkcje silnika wnioskującego, wczytywanie listy spółek, których dane znajdują się w~katalogu z~notowaniami. Tutaj również zdefiniowany jest interfejs użytkownika. Dodatkowo jest tu funkcja uruchamiająca wypakowywanie archiwum ZIP z~notowaniami \\
bachelor.wsk & Tutaj zdefiniowana jest funkcja wczytująca notowania z~pliku i~tworząca z~nich odpowiednią strukturę umieszczaną na liście notowań. W~tej przestrzeni nazw zdefiniowane są również wszystkie wskaźniki wykorzystywane przez system wnioskujący, a~także funkcje pomocnicze do wyliczania wskaźników i~operacji na liście notowań. Jest tu również umieszczona funkcja tworząca listy wszystkich zdefiniowanych wskaźników, a~także funkcje odwołujące się do tych list, które wykorzystywane są w~regułach. \\
bachelor.zip & Zdefiniowana jest tu funkcja do pobierania archiwum ze wskazanego adresu internetowego i~zapisu go na dysku. Znajdują się tu także wszystkie funkcje pomocnicze do rozpakowywania archiwum \\
\end{tabular}

\subsection{Gramatyka reguł}

\paragraph{}
Jedną z funkcjonalności systemu jest parser reguł, które są wczytywane z pliku tekstowego, a następnie na ich podstawie generowany jest kod nowych funkcji, które przekazywane są do mechanizmu wnioskującego. Aby można było stworzyć taki parser najpierw trzeba było zaprojektować gramatykę, zgodnie z którą można zapisywać reguły w pliku tekstowym.