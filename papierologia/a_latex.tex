\chapter{Edycja i~formatowanie pracy}\label{app:edycja}

Pracę należy przygotować korzystając z~systemu składu \LaTeX~(czyt. \textit{latech}). Bardzo dobre wprowadzenie do \LaTeX~stanowi ``The Not So Short Introduction to \LaTeXe'' \cite{Oet11}.

\section{Kwestie techniczne}
Aby móc składać dokumenty z~użyciem \LaTeX~należy go oczywiście najpierw zainstalować. Dostępnych jest wiele dystrybucji \LaTeX. Osobiście korzystam z~TeX Live \cite{texlive}, dostępnego zarówno pod Windowsa jak i~Linuksa. Dobrze jest również zaopatrzyć się w~środowisko do \LaTeX~i~BibTeX (o tym w~sekcji \ref{sec:app:bibliografia}). Użytkownikom Linuksa polecam do tego programy Kile \cite{Kile} oraz KBibTeX \cite{kbibtex}.

\section{Formatowanie}
Należy zachować formatowanie zgodne z~niniejszym szablonem. Preferowany jest druk dwustronny. W~związku z~tym proszę zwrócić szczególną uwagę na położenie szerszego marginesu. Powinien się on oczywiście znajdować od strony bindowania.

Wszystkie ustawienia marginesów, stylu nagłówków, wykorzystanych pakietów etc. znajdują się w~pliku \texttt{praca\_dyplomowa.sty}.

Należy unikać wiszących spójników (zwanych też sierotami). W~tym celu należy stosować po spójnikach twardą spację. W~\LaTeX~uzyskuje się ten efekt poprzez umieszczenie pomiędzy spójnikiem a~następującym po nim słowem znaku tyldy. W~celu ułatwienia tego procesu do szablonu dołączono plik \texttt{korekta.sh}. Jest to prosty skrypt basha -- użytkownicy Windowsa muszą go dostosować do swojego systemu -- który wywołuje polecenia perlowe wstawiające twardą spację po spójnikach we wszystkich plikach z~rozszerzeniem *.tex w~bieżącym katalogu. Przed uruchomieniem skryptu dobrze jest wykonać kopię zapasową.

Przy składaniu pracy przydatny może okazać się tryb draft, który sprawi że zaznaczane będą miejsca w~których tekst nie został prawidłowo złamany. Jeśli \LaTeX~nie potrafi prawidłowo złamać jakiegoś słowa można w~preambule dokumentu użyć polecenia \texttt{hyphenation}.

\section{Bibliografia}\label{sec:app:bibliografia}
Wszystkie źródła informacji, rysunków, danych itd., które zostały wykorzystane przy tworzeniu pracy muszą zostać podane w~bibliografii. Ponadto, wszystkie źródła podane w~bibliografii muszą być cytowane w~tekście. Za każdym razem, kiedy w~pracy pojawia się treść na podstawie jakiegoś tekstu źródłowego czyjegoś autorstwa, oznaczamy takie miejsce przypisem\footnote{\cite{Nie10}, str. 9}. \textbf{Należy pamiętać, że korzystanie ze źródeł bez ich podania w~bibliografii może być podstawą do oskarżenia o~plagiat.}

Należy zwrócić szczególną uwagę na jakość cytowanych źródeł internetowych. Najlepszym rozwiązaniem jest ograniczenie się, na ile to możliwe, do oficjalnych stron projektów. Ponadto, odnośniki do źródeł elektronicznych muszą zawierać pełną ścieżkę do zasobu.

Bibliografię należy przygotować korzystając z~mechanizmów dostarczanych przez LaTeX (patrz rozdział 4.2 w~\cite{Oet11}). Zalecam korzystanie w~tym celu z~BibTeX. BibTeX sam wygeneruje bibliografię na podstawie przygotowanej prostej bazy danych i~zagwarantuje że w~spisie literatury pojawią się tylko te pozycje, które są faktycznie cytowane w~pracy. Pozwoli to zaoszczędzić naprawdę sporo czasu.
