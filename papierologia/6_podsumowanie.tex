\chapter{Podsumowanie i~wnioski}\label{chap:podsumowanie}

\paragraph{}
Podczas pisania pracy udało zrealizować wszystkie cele z~rozdziału \ref{sec:cele_pracy}. Powstał system ekspertowy, którego zadaniem jest prognozowanie przyszłego zachowania rynku finansowego, a~konkretniej spółek giełdowych. System ten udało się zrealizować z~wykorzystaniem paradygmatu programowania funkcyjnego, co~zaowocowało poszerzeniem wiedzy na temat sposobów wytwarzania oprogramowania i~tworzenia kodu źródłowego. Użycie języka Clojure\cite{clj} było bardzo dobrą decyzją. Jest to język mieszany, tzn. jest dialektem najpopularniejszego języka funkcyjnego - Lisp, jednocześnie daje możliwość korzystania z~bibliotek języka Java\cite{java}. Takie połączenie pozwoliło w~łatwy sposób stowrzyć graficzny interfejs użytkownika dla aplikacji. Jednocześnie cała logika systemu mogła zostać zaimplementowana zgodnie z~podejściem funkcyjnym.

\paragraph{}
W~rozdziale \ref{sec:badania} przeprowadzono analizę 15~wybranych spółek giełdowych notowanych na GPW. Spółki wybierane były po~5~spośród największych, średnich i~małych spółek. Wyniki przedstawione w~rozdziale \ref{sec:wyniki_badan} pokazują, że~system w~zdecydowanej większości przypadków poprawnie daje sygnału do zakupu akcji konkretnej spółki. Gorzej wypada w~przypadku wychwytywania sygnałów sprzedaży. Podsumowując jednak całą inwestycję, zarówno zyski z~zakupu akcji jak i~straty podczas sprzedaży, mimo wszystko bilans jest dodatni, co świadczy o~skuteczności działania systemu w~szerokim ujęciu. Z~pewnością niezwykle trudnym zadaniem jest skonstruowanie bazy reguł w~ten sposób, aby~wychwytywać wszystkie sygnały poprawnie. Można nawet pokusić się o~stwierdzenie, że~jest to zadanie niemożliwe do zrealizowania. Spowodowane jest to tym, że~analityk oceniając jakąś spółkę i~analizując dla niej wskaźniki często ufa swoim subiektywnym odczuciom. Nie interpretuje każdego wskaźnika zawsze w~dokładnie ten sam sposób. Ponadto analityk opiera się na swoim doświadczeniu. Przez to opisanie jego pracy w~sposób deklaratywny jest niemożliwe.

\paragraph{}
Praca pokazuje, że~programowanie funkcyjne z~powodzeniem może być zastosowane do tworzenia systemów ekspertowych. Dzięki wykorzystaniu list jako podstawowych struktur do przechowywania analizowanych danych w~bardzo łatwy sposób można te dane przetwarzać. Pozwalają na to zaawansowane, a~zarazem proste w~użyciu, mechanizmy filtrowania list, a~także mapowania funkcji na kolejne elementy w~liście. W~kodzie źródłowym napisanym zgodnie z~podejściem funkcyjnym występuje bardzo dużo wywołań rekurencyjnych, jest to cecha charakterystyczna języków funkcyjnych. Dzięki temu kod pisze się łatwiej, ponieważ rekurencja jest dużo bardziej naturalna dla człowieka. Powoduje to, że~system może powstawać szybciej, a~tworzony kod jest krótszy, ponieważ programista nie zastanawia się jakie kolejne operacje aplikacja ma wykonać, a~prawie że naturalnie opisuje czym poszczególne funkcje/struktury są. Języki funkcyjne pozwalają również na generowanie nowego kodu w~trakcie wykonywania aplikacji, dzięki czemu dobrze napisany system może się rozwijać nawet bez ingerencji programisty w~kod źródłowy. Ponadto kod - funkcje - napisany w~sposób funkcyjny, jeśli jest wystarczająco ogólny, daje się bez większego problemu wykorzystać ponownie w~przyszłości.

\paragraph{}
Z~pewnością istnieje możliwość dalszych badań w~zakresie pracy i~rozwoju aplikacji. Pierwszym krokiem byłoby skalibrowanie reguł systemu tak, aby~wyłapywał precyzyjniej i~w~większej ilości sygnały zarówno do zakupu jak i~sprzedaży akcji. Kolejny etap to zintegrowanie systemu z~notowaniami w~czasie rzeczywistym i~dostosowanie reguł do prognozowania trendów w~ciągu pojedynczej sesji giełdowej. Możliwe jest również rozbudowanie mechanizmu wnioskującego o~heurystyki, które przyspieszyłyby proces wnioskowania, co~byłoby bardzo istotne w~przypadku zleceń zakupu/sprzedaży w~ciągu dnia. Dodatkowo istnieje możliwość rozwoju mechanizmu wyjaśniającego tak, aby~oprócz listy wygenerowanych faktów przedstawiał dokładniejsze wyjaśnienie procesu wnioskowania, np.~w~postaci drzewa ze ścieżką uruchamiania kolejnych reguł i~dodawania nowych faktów do bazy wiedzy.